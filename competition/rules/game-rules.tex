\section {Game Rules}
\label{game-rules}

\begin{enumerate}
\item There will be a maximum of 4 robots in a match.
\item A match lasts 180 seconds.
\item Matches are started and stopped by the Student Robotics radio system\footnote{The Student Robotics radio system is supplied as part of the kit.
 It is part of the power board, and is used for safety cut-off, start-match and stop-match signals.}.
\item There is one type of game that will be played, it is called \textbf{QuackMan}.

At the start of a match, the arena will be populated with 46 tokens.
The tokens are laid out in a pattern of one on the vertex of each square metre in the arena, except for where the three spots where this is too close to the central tower where there are no tokens.
At the top of the tower there is a rubber duck.

\item Robots must start the match within their team's zone.
 Robots will be handed to a [[BLUE SHIRT]] who will place the robot in the appropriate start zone before the beginning of the match.

\item At the end of a match, after all tokens have settled, a team's ``\textbf{game points}'' will be counted.
 These are used to rank teams before competition league points are awarded.

\item Game points are awarded as follows:
\begin{itemize}
\item Red tokens collected are worth 1 game point.
\item Blue tokens collected are worth -5 game points.
\item The duck is worth 100 game points.
\end{itemize}

\item At the end of a game, the team with the \emph{most} game points is awarded 4 points towards the competition league.
 The team with the second most is awarded 3.
 The team with the third most is awarded 2 points, and the team with the fewest game points is awarded 1 point.
 Teams whose robot was not entered into the round, or who were disqualified from the round, will be awarded no points.

\end{enumerate}
