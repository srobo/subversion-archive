\documentclass[a4paper, 12pt]{article}

%\usepackage{savetrees}
\usepackage{graphicx}
\usepackage{subfig}

%code for creating python code snippets
\usepackage{float}
\floatstyle{ruled}
\newfloat{python}{H}{lop}
\floatname{python}{Listing}
%end code for creating python code snippets
\usepackage{verbatim}
\usepackage{moreverb}

\graphicspath{{./images/}}
\title {Student Robotics 2009\\ Programming Reference V2}
\date{\today}
\setcounter{tocdepth}{1}


\begin{document}

\maketitle

\noindent This document details the programming interface for Student Robotics
robots. {\bf This is not intended to be an introduction to the Python programming language.} For more information on programming in Python and about how to bring all this together see the tutorials linked to from the website. 

\section{Directory structure}
The minimum  requirement to create a functioning program using the IDE, is to have a python file called \texttt{robot.py} in a `Project' . Exporting your Project from the IDE will generate a zip file containing your code packaged as a python module. This file can be saved to your memory stick and inserted into your robot.

To make your program easier to read and maintain you can split you code across multiple files within a Project. The export function of the IDE will package the zip file correctly.

\subsection{robot.py}
This file must at least contain the main function for the robot control
algorithm - see Listing \ref{py:main}:

\begin{python}
\begin{verbatimtab}
def main( game, colour ):
        #Where game is either GOLF or SQUIRREL
        #colour is one of the colour constants from above
\end{verbatimtab}
  \caption{\label{py:main}Minimum contents of robot.py} 
\end{python}

\subsection{import}
Before you can control any of the student robotic kit e.g. motors, servos etc. you need to tell python to \textit{import} the Student Robotics interface (functions specific to the SR kit). Listing \ref{py:import} should appear at the top
\begin{python}
\begin{verbatimtab}
#Import the Studnet Robotics Interface	
from sr import * 

#rest of code follows...
\end{verbatimtab}
  \caption{\label{py:import}} 
\end{python}

\section{Main function}
The function \texttt{main} in \texttt{robot.py} is the first part of your code to be executed. It makes the following two variables availible to your code:
\begin{description}
\item[game] The type of competition round being played i.e. Squirrel or Golf
\item[colour]The colour of \textit{your team's} balls.
\end{description}

This information is sent wirelessly to your robot at the start of each round. The game types are available as constants for use in your code, as in Listing \ref{py:gametype}
\begin{python}
\begin{verbatimtab}
#Import the Studnet Robotics Interface	
from sr import * 

def main( game, colour ):
        #Where game is either GOLF or SQUIRREL
        #colour is one of the colour constants from above

	if game == GOLF:
		print "Playing Golf Game..."
		run_golf()	#defined elsewhere

	elif game == SQUIRREL:
		print "Playing Squirrel Game..."
		run_squirrel()	#defined elsewhere
\end{verbatimtab}
  \caption{\label{py:gametype}} 
\end{python}

Similarly the colour constants: \texttt{RED, BLUE, YELLOW, GREEN} will be available for use with the vision methods.


\section{Yielding control and interrupts}
Once the main function is called, the algorithm has control of the robot.  The
algorithm must \texttt{yield} control of the robot once it has
processed any incoming data, and configured any outputs accordingly.
The \texttt{yield} command passes control of the robot over to the
Student Robotics environment. When an event occurs, the environment will pass
control back to the algorithm at the instruction after the \texttt{yield}
statement.  When control is passed back to the
algorithm the \texttt{event} variable has been set to contain information
about the event that just occurred.

\subsection{Timeouts}
The \texttt{yield} statement can be used to wait for a set amount of
time.  For example:

\begin{verbatimtab}
yield 3.4
\end{verbatimtab}

Causes the program to be delayed for 3.4 seconds.

\subsection{Waiting on IO events}
The \texttt{yield} statement can be used to wait for an event to
occur.  For example:

\begin{verbatimtab}
yield io
\end{verbatimtab}

Causes the algorithm to wait until a dio (digital IO) event is received. If you are only interested in certain inputs or conditions, then you can specify this in the yield statement:
\begin{python}
\begin{verbatimtab}
# Wait for input 3 to change digital value
yield io.pin[3]
        
# Wait for input 3 to become digital '1' (threshold 512)
yield io.pin[3] == 1
        
# Wait for input 3 readings to exceed 1V
yield io.apin[1] > 1
        
# Wait for input 2 readings to go below 2.5V
yield io.apin[2] < 2.5
\end{verbatimtab} 
\caption{\label{py:yieldio}Yielding on specified pin inputs}
\end{python}


\subsection{Yielding on Power Board Events}
The power board has a bank of four switches which you can yield on:
\begin{verbatim}
yield power.switch[x] == 0
\end{verbatim}
Where \texttt{x} can take the values: \{0, 1, 2, 3\}.


\subsection{Yielding on Vision Events}
The following command will cause a new image to be captured by the webcam and processed by the slug. Your algorithm will resume when the image has been processed and all the blobs of colours have been extracted from it.
\begin{verbatimtab}
yield vision
\end{verbatimtab}

\subsection{Combinations of events and timeout}
The \texttt{yield} statement can be used to wait on a combination of events and
a timeout, whichever occurs first:

\begin{verbatimtab}
yield io, 5
\end{verbatimtab}

Causes the algorithm to wait until a digital IO event is received or 5 seconds,
whichever comes first.


\subsection{Logic Expressions}
Sometimes you may want to wait for event A to occur AND event B to occur before proceeding. This can be achieved using the `\&' symbol (Logical AND operator) or the And() function:
\begin{python}
\begin{verbatim}
#wait for input pin 3 to be logic 1 and input pin 2 to be logic 0
yield io.pin[3] == 1 & io.pin[2] == 0

#wait for pins 1 and 3 to change, and for pin 2 to be logic 1
yield And(io.pin[1], io.pin[2] == 1, io.pin[3]) 
\end{verbatim}
\caption{\label{py:yieldand}Yielding on multiple events using logical AND}
\end{python}

The \texttt{And()} function can take an unlimited (within reason) number of events. It is functionally equivalent, but can be syntactically easier to read than `\&' operator.

\vspace{12pt}

A logical OR operation can be used to wait for any one of a list of events to occur. It is implemented using the `\texttt{|}' operator or the Or() function:
\begin{python}
\begin{verbatim}
#wait for pin 1 to change or pin 3 to go above 2V:
yield io.pin[1] | io.apin[3]>2

#wait for pin 1 or pin 2 or pin 3 to change:
yield Or(io.pin[1], io.pin[2], io.pin[3])
\end{verbatim}
\caption{\label{py:yieldor}Yielding on multiple events using logical OR}
\end{python}

\noindent\textbf{Note:} Any of the event types discussed above can be used in combination with these logical operators.

\subsection{Calling subroutines}
Control of the robot can be passed to subroutines, allowing the program to be
split up into easier to understand parts.  The \texttt{yield} statement can be
used to pass control of the robot to a subroutine.  When the subroutine returns,
the program continues from the yield statement.  If more information is
passed to yield after the name of a subroutine, then that information is passed
as argument(s) to that subroutine.  For example:

\begin{verbatimtab}
yield findTokens, 3
\end{verbatimtab}

Causes control of the robot to be passed to the findTokens subroutine with the
argument 3.

\section{Events}
\subsection{The \texttt{event} variable}
The \texttt{event} variable is set when control is returned after a
yield. It contains information regarding the event that caused control
to be passed back. It is often necessary to know which type of event has occurred and some further details about the event.
\vspace{12pt}

The \texttt{event} variable will return true when compared to the
event source that led to its creation:

\begin{python}
\begin{verbatimtab}
yield ...

if event == io:
	pass #A io event happened first (pass means do-nothing)
elif event == timeout:
	pass #A timeout event happened first
elif event == vision:
 	pass #A vision event happened first
elif event == power_switch:
	pass #A power switch event happened first
\end{verbatimtab}
\caption{\label{py:events} Student Robotics event types}
\end{python}

\texttt{ io, timeout, vision} \& \texttt{power\_switch} are provided as global types which you can directly access from your robot code.

\subsection{IO Events}
IO events are caused by a change in the digital inputs of the
Joint IO board. After a io event causes a \texttt{yield} statement to
return control of the robot to the control algorithm, the
\texttt{event} variable will be set. The io inputs that have changed
are contained in a dictionary that can be accessed from \texttt{event.pins}: 

\begin{verbatimtab}
yield io.apin[1] > 1.6, (io.pin[2] == 1) & (io.pin[3] == 0)
if 2 in event.io.pins:
    # event.io.vals is an array of the pin values
    # e.g. event.io.vals[2] gives the value of the pin when the event happened
    # the value of event.io.vals[0] is meaningless
    # (and may later throw an error if read at the wrong time)
elif 1 in event.io.pins:
    # event.io.vals[1] is a voltage (float)
\end{verbatimtab}


\subsection{Visual Events}
When the \texttt{vision} event source is passed to the \texttt{yield}
command, the robot will grab an image with its webcam and start to
process that image.  When this processing has finished, the vision
system will raise an event (whether a token is seen or not).  Any
tokens found are placed in the list \texttt{event.vision.blobs}.  Each blob in this
list has a set of properties:  
\begin{description}
\item[blob.x] The centre x coordinate of the blob (measured from left)
\item[blob.y] The centre y coordinate of the blob (measured from top)
\item[blob.mass] The number of pixels that make up the blob
\item[blob.colour] The colour of the blob (RED, BLUE, YELLOW, GREEN)
\item[blob.height] The height in pixels of the blob
\item[blob.width] The width in pixels of the blob
\end{description}

To aid with processing the list of blobs, the following constants are provided:
\begin{description}
\item[VISION\_HEIGHT] The height in pixels of the raw jpg image
\item[VISION\_WIDTH] The width in pixels of the raw jpg image
\end{description}

\begin{python}
\begin{verbatimtab}
yield vision    #This will always return as soon as
                #a frame has been processed

if event == vision:
	for blob in event.vision.blobs: 	#Iterate blob list 
        	if not blob.colour == colour:  	#Ignore Opponents  
                	pass			#blobs.
		else:				#If one of ours,
			pickUpBlob()           	#pick it up!
			break			#break out of loop
\end{verbatimtab}
\caption{\label{py:vision}Example of the vision event and its event properties}
\end{python}


\section{Controlling the robot}
\subsection{Controlling the motors}
The motor controller has two outputs.  The \texttt{setpower} command
sets the direction of each motor and the power delivered to each one.

\begin{description}
\item[setpower(n)] This sets both of the motors to n\% of maximum
power.  $-100<n<100$. Negative values of n drive the motors backwards.
\item[setpower(n, m)] This sets the power delivered to motor 1 to n\%,
  and the power delivered to motor 2 to m\%. $-100<n,m<100$.
\item[readpower(n)] This returns the current percentage power of motor n. Where n=\{0, 1\} 
\end{description}

\vspace{12pt}

\noindent{\bf Extra functions will be added later to support motor feedback }

\subsection{Controlling the PWM board}
The Pulse Width Modulation board is used to control servo motors.  It is
controlled with the \texttt{setpos} function that takes two arguments:

\begin{verbatimtab}
setpos(n, m)
\end{verbatimtab}

This will set PWM board output $0<n<5$ to the position $0<m<100$. \(m = 100\) and \(m = 0\) correspond to the two extreme positions of the servo, and \(m\) can be though of as a percentage.

\vspace{12pt}
To read the current position of any of the servos:
\begin{verbatimtab}
readpos(n)	#n = servo number 
\end{verbatimtab}
If you attempt to read the position of a servo which has not previously been set, readpos() will return \(-1\).

\subsection{Controlling the JointIO Board Outputs}
The outputs and inputs of the JointIO board can be controlled using the following functions:
\begin{verbatimtab}
yield io.apin[1] > 1.6, (io.pin[2] == 1) & (io.pin[3] == 0)
if 2 in event.io.pins:
    # event.io.vals is an array of the pin values
    # e.g. event.io.vals[2] gives the value of the pin when the event happened
    # the value of event.io.vals[0] is meaningless
    # (and may later throw an error if read at the wrong time)
elif 1 in event.io.pins:
    # event.io.vals[1] is a voltage (float)
\end{verbatimtab}

This will set output $0<n<4$ to the value val, which is either 0 or 1.  If value is 1 then the output will be 3.3V, otherwise it will be 0V.

\subsection{Controlling Power Board}
The four LEDs on the Power Board can be controlled using a similar interface to the JointIO:
\begin{verbatimtab}
#similar to jio inputs...
if power.switch[0] == 1:
        
#and similar to jio outputs
power.led[0] = 1
#or
setled(0,1) #where the first argument is the LED number

\end{verbatimtab}

\end{document}
