\documentclass{article}
\usepackage{times}
\usepackage{natbib}
\usepackage{url}
\usepackage{textcomp}
\bibpunct{(}{)}{;}{a}{,}{,}
\begin{document}
\title{Student Robotics Programming Reference}
\date{\today}
\maketitle
\begin{abstract}
This document details the programming interface for Student Robotics
robots.  For more information on programming in Python and about how
to bring all this together see the tutorial. 
\end{abstract}
\section{Directory structure}
The robot source code must be in a zip file named \texttt{robot.zip}.
This zip file must act as a Python module, and so it must contain a
file named \texttt{\_\_init\_\_.py}.  The zip file must contain a
second file named \texttt{robot.py}.  Other \texttt{.py} files can be 
included to improve code organisation.  The checkout function of the
RoboIDE system creates zip files in this format. 

\subsection{\_\_init\_\_.py}
This file is used so Python can identify the zip file as a compressed module.
It should be empty.
\subsection{robot.py}
This file must at least contain the main function for the robot control
algorithm.
\subsection{File structure}
\subsubsection{Imports}
All files containing parts of the robot control algorithm must import the Student
Robotics interface components with: \texttt{from sr import *}.  Other standard
Python modules can be imported as required.
\section{Main function}
In \texttt{robot.py} there must be a function with the form:

\begin{verbatim}
def main(corner, colour, game):
\end{verbatim}

When the robot is started, this function will be called.
\section{Yielding control and interrupts}
Once the main function is called, the algorithm has control of the robot.  The
algorithm must \texttt{yield} control of the robot once it has
processed any incoming data, and configured any outputs accordingly.
The \texttt{yield} command passes control of the robot over to the
Student Robotics environment which will pass control back to the instruction
after the \texttt{yield} statement.  When control is passed back to the
algorithm the \texttt{event} variable is set to contain information
about the event that just occurred.

\subsection{Using \texttt{yield}}
\subsubsection{Timeouts}
The \texttt{yield} statement can be used to wait for a set amount of
time.  For example:

\begin{verbatim}
yield 4
\end{verbatim}

Causes the program to be delayed for 4 seconds.

\subsubsection{Waiting on events}
The \texttt{yield} statement can be used to wait for an event to
occur.  For example:

\begin{verbatim}
yield dio
\end{verbatim}

Causes the algorithm to wait until a dio (digital IO) event is received.

\subsubsection{Combinations of events and timeout}
The \texttt{yield} statement can be used to wait on a combination of events and
a timeout, whichever occurs first:

\begin{verbatim}
yield dio, 5
\end{verbatim}

Causes the algorithm to wait until a dio event is received or 5 seconds,
whichever comes first.

\subsubsection{Calling subroutines}
Control of the robot can be passed to subroutines, allowing the program to be
split up into easier to understand parts.  The \texttt{yield} statement can be
used to pass control of the robot to a subroutine.  When the subroutine returns,
the program continues from the yield statement.  If more information is
passed to yield after the name of a subroutine, then that information is passed
as argument(s) to that subroutine.  For example:

\begin{verbatim}
yield findTokens, 3
\end{verbatim}

Causes control of the robot to be passed to the findTokens subroutine with the
argument 3.

\subsection{The \texttt{event} variable}
The \texttt{event} variable is set when control is returned after a
yield. It contains information regarding the event that caused control
to be passed back.

\subsubsection{Finding the event source}
The \texttt{event} variable will return true when compared to the
event source that led to its creation:

\begin{verbatim}
yield dio, 5 #Wait until a dio event or 5 seconds passes

if event == dio:
    pass #A dio event happened first (pass means do-nothing)
else:
    pass #A timeout is the only other option
\end{verbatim}

\subsection{Digital IO Events}
Digital IO events are caused by a change in the digital inputs of the
Joint IO board. After a dio event causes a \texttt{yield} statement to
return control of the robot to the control algorithm, the
\texttt{event} variable will be set. The dio inputs that have changed
are contained in a dictionary that can be accessed from \texttt{dio.pins}: 

\begin{verbatim}
yield dio, 5

if event == dio:               #Is it a dio event?
    if 0 in event.pins:        #Has input 0 been changed?
        if event.pins[0] == 1: #Is input 0 now a 1?
            pass
\end{verbatim}

\subsection{Visual Events}
When the \texttt{vision} event source is passed to the \texttt{yield}
command, the robot will grab an image with its webcam and start to
process that image.  When this processing has finished, the vision
system will raise an event (whether a token is seen or not).  Any
tokens found are placed in the list \texttt{event.blobs}.  Each blob in this
list has a set of properties:  
\begin{description}
\item[colour] The colour of the blob
\item[mass] The number of pixels that make up the blob
\item[centrex] The number of pixels between the left of the image and
  the centre of the blob.
\item[centrey] The number of pixels down from the top of the image to the
  centre of the blob.
\end{description}

\begin{verbatim}
yield vision    #This will always return as soon as
                #a frame has been processed

for blob in event.blobs:    #Go through each blob in 
                            #turn. This does nothing if
                            #there are no blobs
    if blob.colour == 0:    #For each blob, check its colour
        pass
\end{verbatim}

\section{Controlling the robot}
\subsection{Controlling the motors}
The motor controller has two outputs.  The \texttt{setspeed} command
sets the direction of each motor and the power delivered to each one.

\begin{description}
\item[setspeed(n)] This sets both of the motors to n\% of maximum
power.  $-100<n<100$. Negative values of n drive the motors backwards.
\item[setspeed(n, m)] This sets the power delivered to motor 1 to n\%,
  and the power delivered to motor 2 to m\%. $-100<n,m<100$.
\end{description}

\subsection{Controlling the PWM board}
The Pulse Width Modulation board is used to control servo motors.  It is
controlled with the \texttt{pwm} function that takes two arguments:

\begin{verbatim}
pwm(n, m)
\end{verbatim}

This will set PWM board output $0<n<5$ to the position $0<m<100$.
\end{document}
